\RequirePackage{fix-cm}
\documentclass[9pt]{article}
\usepackage[fontsize=8.5pt]{scrextend}
\usepackage[top=0.5in, left=0.3in, right=0.3in, bottom =0.5in]{geometry}
\usepackage{booktabs}
\usepackage{fancyhdr}
\usepackage[utf8]{inputenc}
\usepackage{ctable}
\newcommand{\changefont}{\fontsize{8}{10}\selectfont}
\pagestyle{fancy}
\fancyhf{}
\rfoot[LE,RO]{\changefont \slshape Curriculum Vitae}
\lfoot[RE,LO]{\changefont \slshape James Ovenden}
\cfoot{\thepage}
\renewcommand{\footrulewidth}{1pt}
\renewcommand{\headrulewidth}{0pt}
\hyphenpenalty=10000
\setlength{\parindent}{0cm}
\usepackage{enumitem}
\pagenumbering{gobble}

\begin{document}
%TITLE%%%%%%%%%%%%%%%%%%%%%%%%%%%%%%%%%%%%%%%%%%%%%%%%%%%%%%%%%
\begin{center}
\bfseries{James Ovenden}\\
\normalfont{07874 770 118, jl.ovenden@yahoo.co.uk}\\
\end{center}
%PROFILE AND RESEARCH%%%%%%%%%%%%%%%%%%%%%%%%%%%%%%%%%%%%%%%%%%%%%%%%%%%%%%%%%
\rule[0.1cm]{\textwidth}{0.05cm}\\
\bfseries{Profile and Research Interests}\\
\vspace{0cm}\\
\normalfont{A driven, enthusiastic, and hard-working physicist from Ipswich: I obtained a 2:1 in a Theoretical Physics Masters Degree at the University of Leeds. My research interests include quantum mechanics and other aspects of particle physics.\\ 
My studies involved applying the topology and geometry of electromagnetism to the magnetic monopole in my dissertation and considering the application of quantum mechanics to technology and computing.}\\
%EMPLOYMENT HISTORY%%%%%%%%%%%%%%%%%%%%%%%%%%%%%%%%%%%%%%%%%%%%%%%%%%%%%%%%%
\rule[0.1cm]{\textwidth}{0.04cm}
\bfseries{Employment History}\\
\vspace{0.1cm}\\
\normalfont{Feb 2023 - July 2023} \hspace{77pt} \bfseries{Maths Support Assistant, The Royal Hospital School} \hspace{55pt} \normalfont{Holbrook}\\
\vspace{0cm}\\
My role was a mix of teaching assistant, support and cover staff:
\begin{itemize}[itemsep=0mm, parsep=0pt]
\item Using my excellent communication skills with pupils and colleagues and my assets as a tutor to intermittently work with students in a one-to-one manner to more successfully identify problems with depth and intuition and assist them as such.
\item Effective management of the classroom with the separation of uncooperative students, removal of potential distractions and acquisition of necessary in-class items.
\item My work in the school saw students' grades evidently improve. I was continuously given positive feedback from my peers, too.
\end{itemize}
\normalfont{Dec 2020 -} \hspace{119pt} \bfseries{STEM subject Tutor for MyTutor and Virtual Tutoring} \hspace{160pt} \normalfont{Kesgrave}\\
\vspace{0cm}\\
I conduct lessons over the Internet with students from all over the world at GCSE, A-level and degree-level. I have over 1000 hours of tutoring experience and have received numerous positive reviews averaging a feedback score of 4.9/5 with many pieces positive feedback. At my peak, I would work with around 15 regular students per week in private lessons and provide:\\
\begin{itemize}[itemsep=0mm, parsep=0pt]
\item Clear and effective communication with students either verbally or via an online whiteboard system. These online systems were valuable when teaching a student with autism and ADHD, who achieved far above their initial goals.
\item Producing learning resources, either for communication of concepts or for sample questions and solutions. 
\item Lesson planning incorporates resources and allows for efficacy in their use. Planning also makes for more successful lesson time management.
\end{itemize}
\normalfont{Sep 2012 - Jan 2018} \hspace{80pt} \bfseries{Golf Department Assistant, Ufford Park Golf Club} \hspace{70pt} \normalfont{Melton}\\
\vspace{0cm}\\
I was fortunate enough to gain work experience at a young age. I worked as part of a team where we each had responsibilities in the Golf reception. I feel I developed skills that would be useful in any environment, such as:
\begin{itemize}[itemsep=0mm, parsep=0pt]
\item Using my communication skills with customers (I’d interact with around 70 customers daily) who enquired about services, either in person or over the phone. 
\item Working as a team to complete daily tasks (for example, restocking drinks and confectionery items or responding to emails daily) for the upkeep of the department.
\end{itemize}
%UNI EDUCATION%%%%%%%%%%%%%%%%%%%%%%%%%%%%%%%%%%%%%%%%%%%%%%%%%%%%%%%%%
\rule[0.1cm]{\textwidth}{0.04cm}
\bfseries{Education}\\
\vspace{0.1cm}\\
\normalfont{Sep 2016 - Jun 2020} \hspace{73pt} \bfseries{University of Leeds, MPhys Theoretical Physics} \hspace{87pt} \normalfont{Leeds}\\
\vspace{0cm}\\	
\vspace{0cm}\\
\bfseries{Year 4 research project---The Search For Magnetic Monopoles, 77\%}\\
\vspace{0cm}\\
\normalfont{In my Year 4 research project, The Search For Magnetic Monopoles, the first assignment was a literature review. I started by considering monopole-induced symmetry in Maxwell's equations and monopole-created Dirac charge quantisation condition. I also looked at monopoles in cosmology with monopole production due to symmetry breaking in the Higgs Field, I posited how magnetic charge arises from this. \\
\\
Over time my work here took me through the mathematical foundations and the physical applications of principal fibre bundles. I then analysed these with de Rham cohomology, Chern characteristic classes and Chern numbers. I could then determine why the magnetic charge is topological. This project gave me skills that would be transferrable to any professional research environment, such as:
\begin{itemize}[itemsep=0mm, parsep=0pt]
\item Time management and general independent studying skills
\item Researching appropriate scientific papers and textbooks before extracting relevant information.
\item Effective teamwork and communication with my personal tutor when working through scientific and mathematical concepts.
\end{itemize}
The research process here has heavily boosted my interest in formalisms and results from particle physics. I am now determined to take my career in research in science further.\\}\\
\bfseries{Relevant Modules:}\\
\vspace{0cm}\\
\bfseries{Year 4: Quantum Field Theory, 89\% and General Relativity, 69\%}\\
\vspace{0cm}\\
\normalfont{Topics here included looking at functionals in the build-up to classical field theory. I also learned about Interacting scalar field theory, Feynman rules and diagrams before being introduced to canonical quantisation and the path-integral approach.\\}\\
\normalfont{General Relativity started by adding gravity to special relativity with local inertial frames and tensor transformations. I continued this with metrics, geodesics, Riemannian manifolds, and parallel transport before considering the Euler-Lagrange equation.\\}
\vspace{0cm}\\
\bfseries{Year 3: Advanced Quantum Mechanics, 60\%}\\
\vspace{0cm}\\
\normalfont{This module piqued my interest with some more introductory quantum concepts such as the Schrodinger equation in Cartesian coordinates, Matrix mechanics (and Pauli spin matrices) and the Dirac formalism.\\}
\vspace{0cm}\\
\bfseries{Year 3: Theoretical Elementary Particle Physics, 65\% and Quantum Photonics, 61\%}\\
\vspace{0cm}\\
\normalfont{Particle Physics looked at the results of quantum field theory, such as the Klein-Gordon, Proca, and Dirac equations. Quantum Photonics studied the Schrodinger, Heisenberg, and Interacting pictures.\\}
\vspace{0cm}\\
\bfseries{Year 2: Computing 2, 95\% and Year 1: Computing 1, 86\%}\\
\vspace{0cm}\\
\normalfont{In these, I used Python to solve example day-to-day problems and, later, to analyse data from an x-ray diffraction experiment to determine the lattice parameter of a metal alloy.\\}
\vspace{0cm}\\
%SCHOOL EDUCATION%%%%%%%%%%%%%%%%%%%%%%%%%%%%%%%%%%%%%%%%%%%%%%%%%%%%%%%%%
\rule[0.1cm]{\textwidth}{0.05cm}
\normalfont{Sep 2013 - Jul 2016} \hspace{75pt} \bfseries{Kesgrave High School, A-Levels} \hspace{167pt}\\
\vspace{0cm}\\
\bfseries{A-Levels: } \normalfont{Mechanical Mathematics A, Physics B, ICT B, Geography C}\\
\bfseries{AS-Levels: } \normalfont{Statistical Mathematics A, Further Mathematics C, Computing D}\\
\vspace{0cm}\\
\normalfont{Sep 2010 - Jul 2013} \hspace{75pt} \bfseries{Kesgrave High School, Key Stage 4} \hspace{150pt}\\
\vspace{0cm}\\
\bfseries{GCSEs: } \normalfont{2 A*'s, 4 A's, 3 B's}\hspace{270pt}\bfseries{BTECs: } \normalfont{Distinction, Merit}\\
%SKILLS%%%%%%%%%%%%%%%%%%%%%%%%%%%%%%%%%%%%%%%%%%%%%%%%%%%%%%%%%
\rule[0.1cm]{\textwidth}{0.04cm}
\bfseries{Skills}\\
\vspace{0cm}\\
\bfseries{Laboratory Research: }\normalfont{Proficient. Lab modules in the early part of my degree allowed me to pick up valuable skills such as calculating and minimising uncertainties and general use of apparatus.}\\
\vspace{0cm}\\
\bfseries{Academic Writing: }\normalfont{Proficient. My time at Leeds introduced me to different forms of academic writing with positive results, culminating in a final report at the end of my project. I received great credit for the writing in my dissertation. It was also deemed very professionally presented.}\\
\vspace{0cm}\\
\bfseries{Working with others: }\normalfont{Excellent teamwork and communication skills. I love working with students in my current occupation and liked conspiring with peers in my previous roles. Cooperating with others from differing backgrounds at University and collaborating with students from all over the world in my current occupation is incredibly fulfilling.}\\
\vspace{0cm}\\
\bfseries{Teaching: }\normalfont{Accomplished—Helping students with physics and mathematics, the companies I work for have trained me in many aspects such as lesson planning, production of resources, and safeguarding with a ’safeguarding children level 1’ course. I have amassed over 1000 contact hours and numerous positive results and reviews from students: from GCSE to 2nd-year University-level. Teaching, with tasks such as lesson planning and completion of lesson reports, have also taught me to manage my time more effectively.}\\
\vspace{0cm}\\
\bfseries{Presenting: }\normalfont{Accomplished. I’ve had plenty of experience presenting concepts and ideas throughout my education, and it is something in which I have always excelled. I have also implemented it in my current tutoring role, where I have found an efficient way to produce fluent and coherent lessons.}\\
\vspace{0cm}\\
\bfseries{Programming: }
\normalfont{\begin{itemize}[itemsep=0mm, parsep=0pt]
\item Python: Accomplished
\item LaTeX: Accomplished
\item Java: Good
\end{itemize}}
\vspace{0cm}
\bfseries{Computer Skills: }\normalfont{Windows: Excellent; MacIOS: Excellent}\\
%GENERAL INTERESTS%%%%%%%%%%%%%%%%%%%%%%%%%%%%%%%%%%%%%%%%%%%%%%%%%%%%%%%%%
\rule[0.1cm]{\textwidth}{0.05cm}
\bfseries{General Interests}\\
\vspace{0cm}\\
\normalfont{Physics has always fascinated me: I subscribe to many online discussion groups and mailbases. Theories of everything piqued my interest in theoretical physics, such as M–theory. Understanding just some of the mathematics behind M–theory is a big reason for my educational path.\\
I find programming offers fun and engaging analytical tasks. I like to create programs to analyse data in my free time.\\
I am also an avid reader of current affairs. I’m sure to keep up to date with news in my academic area, especially in this current climate with the Muon g-2 experiment.}
\vspace{0.1cm}\\
%%%%%%%%%%%%%%%%%%%%%%%%%%%%%%%%%%%%%%%%%%%%%%%%%%%%%%%%%%
\rule[0.1cm]{\textwidth}{0.04cm}
\bfseries{Referees}\\
\vspace{0cm}\\
\normalfont{Available upon request.}
\end{document}