\documentclass[10pt]{article}
\usepackage[margin=0.6in]{geometry}
\usepackage{booktabs}
\usepackage{fancyhdr}
\usepackage[utf8]{inputenc}
\usepackage{ctable}
\newcommand{\changefont}{\fontsize{9}{10}\selectfont}
\pagestyle{fancy}
\fancyhf{}
\rfoot[LE,RO]{\changefont \slshape Curriculum Vitae}
\lfoot[RE,LO]{\changefont \slshape James Ovenden}
\cfoot{\thepage}
\renewcommand{\footrulewidth}{1pt}
\renewcommand{\headrulewidth}{0pt}
\hyphenpenalty=10000
\setlength{\parindent}{0cm}
\usepackage{enumitem}
\pagenumbering{gobble}

\begin{document}
%TITLE%%%%%%%%%%%%%%%%%%%%%%%%%%%%%%%%%%%%%%%%%%%%%%%%%%%%%%%%%
\begin{center}
\bfseries{James Ovenden}\\
\normalfont{07874 770 118, jl.ovenden@yahoo.co.uk}\\
\end{center}
%PROFILE AND RESEARCH%%%%%%%%%%%%%%%%%%%%%%%%%%%%%%%%%%%%%%%%%%%%%%%%%%%%%%%%%
\rule[0.1cm]{\textwidth}{0.05cm}\\
\bfseries{Profile and Research Interests}\\
\vspace{0cm}\\
\normalfont{A driven, enthusiastic, and hard-working physicist from Ipswich: I obtained a 2:1 in a Theoretical Physics Masters Degree at the University of Leeds. My research interests include quantum mechanics and other aspects of particle physics.\\ 
My studies involved applying the topology and geometry of electromagnetism to the magnetic monopole in my dissertation and considering the application of quantum mechanics to technology and computing. The experience of pursuing this sort of research further with the University of Southampton is a prospect that excites me greatly.}\\
%UNI EDUCATION%%%%%%%%%%%%%%%%%%%%%%%%%%%%%%%%%%%%%%%%%%%%%%%%%%%%%%%%%
\rule[0.1cm]{\textwidth}{0.04cm}
\bfseries{Education}\\
\vspace{0.1cm}\\
\normalfont{Sep 2016 - Jun 2020} \hspace{73pt} \bfseries{University of Leeds, MPhys Theoretical Physics} \hspace{87pt} \normalfont{Leeds}\\
\vspace{0cm}\\	
\bfseries{Relevant Modules, Year 4:}\\
\vspace{0cm}\\
\bfseries{Quantum Field Theory, 89\%}\\
\vspace{0cm}\\
\normalfont{Quantum field theory captured what I enjoy so much about theoretical physics. This passion for the topic shows in my overall mark of 89\%. Some of the most important topics covered include:
\begin{itemize}[itemsep=0mm, parsep=0pt]
\item Functionals in the build-up to classical field theory
\item Introduction to canonical quantisation 
\item Working with ladder and field operators
\item The LSZ formula, Wick's theorem, and Dyson-Wick expansion
\item  Interacting scalar field theory, Feynman rules and diagrams
\item The path-integral approach
\item Renormalisation with bare and renormalised perturbation theory. Introducing further Feynman rules. 
\end{itemize}
Quantum field theory gave me a great foundation in one of the most seminal theories in modern particle physics.  Expanding upon my knowledge here is paramount to my goals as a scientist.\\} 
\vspace{0cm}\\
\bfseries{General Relativity, 69\%}\\
\vspace{0cm}\\
\normalfont{General relativity is another year 4 module that allowed me to learn about many tools and techniques that are now so necessary in modern theoretical physics and cosmology:}
\begin{itemize}[itemsep=0mm, parsep=0pt]
\item Adding gravity to special relativity with local inertial frames and tensor transformations
\item Continuing this with metrics, geodesics, Riemannian manifolds, and parallel transport
\item Further looks at Riemannian geometry with the Riemann and Ricci tensors
\item Formalising the field equations before looking at the Euler-Lagrange equations
\item Looking at the Robertson-Walker and Schwarzchild metrics
\item Applying general relativity to cosmology.
\end{itemize}
\normalfont{Learning general relativity at University was always an important goal of mine. I would love to apply what I learned here in a professional environment.\\} 
\vspace{0cm}\\
\bfseries{Relevant Modules, Year 3:}\\
\vspace{0cm}\\
\bfseries{Advanced Quantum Mechanics, 60\%}\\
\vspace{0cm}\\
\normalfont{I was able to obtain 60\% in Advanced Quantum Mechanics. This module sparked further my interest with some more introductory quantum concepts: 
\begin{itemize}[itemsep=0mm, parsep=0pt]
\item The Schrodinger equation in Cartesian coordinates
\item Matrix mechanics and the Dirac formalism
\item The variational method and application of this to physical
\item Pauli spin matrices and solving associated problems
\end{itemize}
\vspace{1cm}
\bfseries{Theoretical Elementary Particle Physics, 65\% and Quantum Photonics, 61\%}\\
\vspace{0cm}\\
\normalfont{Particle Physics looked at the results of quantum field theory such as the Klein-Gordon, Proca, and Dirac equations---where the latter introduced the concept of spinors. It also sparked my love of Quantum Field Theory. Quantum Photonics studied the Schrodinger, Heisenberg, and Interacting pictures. I obtained 65\% and 61\% respectively in these courses.\\}
\vspace{0cm}\\
\bfseries{Year 4 research project---The Search For Magnetic Monopoles, 77\%}\\
\vspace{0cm}\\
\normalfont{My Year 4 research project, The Search For Magnetic Monopoles, allowed me to delve into Physics outside taught in University modules. The first major assignment in this project was a literature review, allowing me to scratch the surface of all sorts of areas involving the magnetic monopole:
\begin{itemize}[itemsep=0mm, parsep=0pt]
\item The monopole-induced symmetry in Maxwell's equations
\item Monopole-created Dirac charge quantisation condition
\item Monopoles in Cosmology
\item Monopole production due to symmetry breaking in the Higgs Field
\item EM-Higgs field coupling: positing how magnetic charge arises from this. 
\end{itemize}
I've used this project as an opportunity to study the topological aspects and geometry of particle physics and to ask the question of how this theoretical `topological defect' is of magnetic nature. My work here has taken me through:
\begin{itemize}[itemsep=0mm, parsep=0pt]
\item Looking at homeomorphisms and homotopies
\item Researching differentiable manifolds and Lie groups
\item Looking at principal fibre bundles
\item Looking at connections and curvature on fibre bundles
\item Applying this, specifically, to the Hopf bundle
\item de Rahm Cohomology allowed me to develop my knowledge of Chern characteristic classes
\item Chern numbers related the mathematical to the physical and determined why the magnetic charge is topological. 
\end{itemize}
I loved learning different mathematical techniques and concepts, but this project gave me skills that would be transferrable to any professional research environment, such as:
\begin{itemize}[itemsep=0mm, parsep=0pt]
\item Time management and general independent studying skills
\item Researching appropriate scientific papers and textbooks before extracting relevant information.
\end{itemize}
The research process here has served to heavily boost my interest in formalisms and results from particle physics. I am now determined to take my findings in the field further.}\\
%SCHOOL EDUCATION%%%%%%%%%%%%%%%%%%%%%%%%%%%%%%%%%%%%%%%%%%%%%%%%%%%%%%%%%
\vspace{0cm}\\
\normalfont{Sep 2013 - Jul 2016} \hspace{75pt} \bfseries{Kesgrave High School, A-Levels} \hspace{167pt}\\
\vspace{0cm}\\
\bfseries{A-Levels: } \normalfont{Mechanical Mathematics A, Physics B, ICT B, Geography C}\\
\bfseries{AS-Levels: } \normalfont{Statistical Mathematics A, Further Mathematics C, Computing D}\\
\vspace{0cm}\\
\normalfont{Sep 2010 - Jul 2013} \hspace{75pt} \bfseries{Kesgrave High School, Key Stage 4} \hspace{150pt}\\
\vspace{0cm}\\
\bfseries{GCSEs: } \normalfont{2 A*'s, 4 A's, 3 B's}\hspace{270pt}\bfseries{BTECs: } \normalfont{Distinction, Merit}\\
%SKILLS%%%%%%%%%%%%%%%%%%%%%%%%%%%%%%%%%%%%%%%%%%%%%%%%%%%%%%%%%
\rule[0.1cm]{\textwidth}{0.04cm}
\bfseries{Skills}\\
\vspace{0cm}\\
\bfseries{Laboratory Research: }\normalfont{Proficient. Lab modules in the early part of my degree allowed me to pick up valuable skills such as calculating and minimising uncertainties and general use of apparatus.}\\
\vspace{0cm}\\
\bfseries{Academic Writing: }\normalfont{Proficient. My time at Leeds introduced me to different forms of academic writing with positive results, culminating in a final report at the end of my project. I received great credit for the writing of my dissertation, with it also deemed very professionally presented.}\\
\vspace{0cm}\\
\bfseries{Working with others: }\normalfont{Excellent teamwork and communication skills. I love working with students in my current occupation and liked conspiring with peers in my previous roles. Cooperating with others from differing backgrounds at University and collaborating with students from all over the world in my current occupation is incredibly fulfilling.}\\
\vspace{0cm}\\
\bfseries{Teaching: }\normalfont{Accomplished—Helping students with physics and mathematics, the companies I work for have given me training in many aspects such as lesson planning, production of resources, and safeguarding. The latter resulted in me completing the ’safeguarding children level 1’ course. I have amassed over 1000 contact hours and numerous positive results and reviews from a range of students: from GCSE to 2nd-year University-level. It’s hugely rewarding when, using my intuition to think outside the box, I can enable a student to understand a challenging topic. Teaching, with tasks such as lesson planning and filling out lesson reports, has also taught me to manage my time more effectively.}\\
\vspace{0cm}\\
\bfseries{Presenting: }\normalfont{Accomplished. I’ve had plenty of experience presenting concepts and ideas throughout my education, and it is something in which I have always excelled. I have also implemented it in my current tutoring role, where I have found an efficient way to produce fluent and coherent lessons for students.}\\
\vspace{0cm}\\
\bfseries{Programming: }
\begin{itemize}[itemsep=0mm, parsep=0pt]
\item \normalfont{Python: Accomplished. As shown in my grades of 95\% and 86\% in my University computing modules. In these, I used Python to solve problems and analyse experimental data.} 
\item LaTeX: Accomplished
\item Java: Good
\end{itemize}
\vspace{0cm}
\bfseries{Computer Skills: }\normalfont{Windows: Excellent; MacIOS: Excellent}\\
%EMPLOYMENT HISTORY%%%%%%%%%%%%%%%%%%%%%%%%%%%%%%%%%%%%%%%%%%%%%%%%%%%%%%%%%
\rule[0.1cm]{\textwidth}{0.04cm}
\bfseries{Employment History}\\
\vspace{0.1cm}\\
\normalfont{Feb 2023 - July 2023} \hspace{77pt} \bfseries{Maths Support Assistant, The Royal Hospital School} \hspace{55pt} \normalfont{Holbrook}\\
\vspace{0cm}\\
I took the opportunity to grow my existing skills in a working environment by beginning with the Royal Hospital School in Holbrook, Suffolk. My role is a mix of teaching assistant, support and cover staff. I'd interact with around 5 separate classes of approximately 20 per day from years 7 to 13.
\begin{itemize}[itemsep=0mm, parsep=0pt]
\item Using my excellent communication skills with pupils and colleagues and my assets as a tutor to intermittently work with students in a one-to-one manner to more successfully identify problems with depth and intuition and assist them as such.
\item Using my initiative to identify and solve problems to manage the classroom such as separation of uncooperative students, removal of potential distractions and acquisition of necessary in-class items.
\end{itemize}
\normalfont{Dec 2020 -} \hspace{119pt} \bfseries{STEM subject Tutor for MyTutor and Virtual Tutoring} \hspace{160pt} \normalfont{Kesgrave}\\
\vspace{0cm}\\
I conduct lessons over the Internet with students from all over the world at GCSE, A-level and degree-level. I have over 1000 hours of tutoring experience and have received numerous positive reviews averaging a feedback score of 4.9/5. At my peak, I would work with around 15 regular students per week in private lessons.
\\
\vspace{0cm}\\
My key responsibilities:
\begin{itemize}[itemsep=0mm, parsep=0pt]
\item Clear and effective communication with students. This can be done verbally or with online whiteboard systems for students who are responsive to visual aids. These online systems were valuable when teaching a student with autism and ADHD.
\item Producing learning resources, some of which help in the communication of concepts. Many will also be example questions and solutions as I believe this is the best way to learn physics and maths.
\item Lesson planning incorporates resources and allows for efficacy in their use. Planning also makes for more successful lesson time management.
\end{itemize}
%GENERAL INTERESTS%%%%%%%%%%%%%%%%%%%%%%%%%%%%%%%%%%%%%%%%%%%%%%%%%%%%%%%%%
\rule[0.1cm]{\textwidth}{0.05cm}
\bfseries{General Interests}\\
\vspace{0cm}\\
\normalfont{Physics has always fascinated me: I subscribe to many online discussion groups and mailbases. One of the theories that got me interested in theoretical physics was that of M–theory. Understanding at least some of the mathematics behind M–theory is a big reason for my educational path.\\
I find programming offers fun and engaging analytical tasks. I like to create programs to analyse data in my free time.\\
I am also an avid reader of current affairs. I’m sure to keep up to date with news in my academic area, especially in this current climate with the Muon g-2 experiment.}
\vspace{0.1cm}\\
%%%%%%%%%%%%%%%%%%%%%%%%%%%%%%%%%%%%%%%%%%%%%%%%%%%%%%%%%%
\rule[0.1cm]{\textwidth}{0.04cm}
\bfseries{Referees}\\
\vspace{0cm}\\
\normalfont{Available upon request.}
\end{document}